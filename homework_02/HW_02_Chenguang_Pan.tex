% Options for packages loaded elsewhere
\PassOptionsToPackage{unicode}{hyperref}
\PassOptionsToPackage{hyphens}{url}
%
\documentclass[
]{article}
\usepackage{amsmath,amssymb}
\usepackage{lmodern}
\usepackage{iftex}
\ifPDFTeX
  \usepackage[T1]{fontenc}
  \usepackage[utf8]{inputenc}
  \usepackage{textcomp} % provide euro and other symbols
\else % if luatex or xetex
  \usepackage{unicode-math}
  \defaultfontfeatures{Scale=MatchLowercase}
  \defaultfontfeatures[\rmfamily]{Ligatures=TeX,Scale=1}
\fi
% Use upquote if available, for straight quotes in verbatim environments
\IfFileExists{upquote.sty}{\usepackage{upquote}}{}
\IfFileExists{microtype.sty}{% use microtype if available
  \usepackage[]{microtype}
  \UseMicrotypeSet[protrusion]{basicmath} % disable protrusion for tt fonts
}{}
\makeatletter
\@ifundefined{KOMAClassName}{% if non-KOMA class
  \IfFileExists{parskip.sty}{%
    \usepackage{parskip}
  }{% else
    \setlength{\parindent}{0pt}
    \setlength{\parskip}{6pt plus 2pt minus 1pt}}
}{% if KOMA class
  \KOMAoptions{parskip=half}}
\makeatother
\usepackage{xcolor}
\usepackage[margin=1in]{geometry}
\usepackage{color}
\usepackage{fancyvrb}
\newcommand{\VerbBar}{|}
\newcommand{\VERB}{\Verb[commandchars=\\\{\}]}
\DefineVerbatimEnvironment{Highlighting}{Verbatim}{commandchars=\\\{\}}
% Add ',fontsize=\small' for more characters per line
\usepackage{framed}
\definecolor{shadecolor}{RGB}{248,248,248}
\newenvironment{Shaded}{\begin{snugshade}}{\end{snugshade}}
\newcommand{\AlertTok}[1]{\textcolor[rgb]{0.94,0.16,0.16}{#1}}
\newcommand{\AnnotationTok}[1]{\textcolor[rgb]{0.56,0.35,0.01}{\textbf{\textit{#1}}}}
\newcommand{\AttributeTok}[1]{\textcolor[rgb]{0.77,0.63,0.00}{#1}}
\newcommand{\BaseNTok}[1]{\textcolor[rgb]{0.00,0.00,0.81}{#1}}
\newcommand{\BuiltInTok}[1]{#1}
\newcommand{\CharTok}[1]{\textcolor[rgb]{0.31,0.60,0.02}{#1}}
\newcommand{\CommentTok}[1]{\textcolor[rgb]{0.56,0.35,0.01}{\textit{#1}}}
\newcommand{\CommentVarTok}[1]{\textcolor[rgb]{0.56,0.35,0.01}{\textbf{\textit{#1}}}}
\newcommand{\ConstantTok}[1]{\textcolor[rgb]{0.00,0.00,0.00}{#1}}
\newcommand{\ControlFlowTok}[1]{\textcolor[rgb]{0.13,0.29,0.53}{\textbf{#1}}}
\newcommand{\DataTypeTok}[1]{\textcolor[rgb]{0.13,0.29,0.53}{#1}}
\newcommand{\DecValTok}[1]{\textcolor[rgb]{0.00,0.00,0.81}{#1}}
\newcommand{\DocumentationTok}[1]{\textcolor[rgb]{0.56,0.35,0.01}{\textbf{\textit{#1}}}}
\newcommand{\ErrorTok}[1]{\textcolor[rgb]{0.64,0.00,0.00}{\textbf{#1}}}
\newcommand{\ExtensionTok}[1]{#1}
\newcommand{\FloatTok}[1]{\textcolor[rgb]{0.00,0.00,0.81}{#1}}
\newcommand{\FunctionTok}[1]{\textcolor[rgb]{0.00,0.00,0.00}{#1}}
\newcommand{\ImportTok}[1]{#1}
\newcommand{\InformationTok}[1]{\textcolor[rgb]{0.56,0.35,0.01}{\textbf{\textit{#1}}}}
\newcommand{\KeywordTok}[1]{\textcolor[rgb]{0.13,0.29,0.53}{\textbf{#1}}}
\newcommand{\NormalTok}[1]{#1}
\newcommand{\OperatorTok}[1]{\textcolor[rgb]{0.81,0.36,0.00}{\textbf{#1}}}
\newcommand{\OtherTok}[1]{\textcolor[rgb]{0.56,0.35,0.01}{#1}}
\newcommand{\PreprocessorTok}[1]{\textcolor[rgb]{0.56,0.35,0.01}{\textit{#1}}}
\newcommand{\RegionMarkerTok}[1]{#1}
\newcommand{\SpecialCharTok}[1]{\textcolor[rgb]{0.00,0.00,0.00}{#1}}
\newcommand{\SpecialStringTok}[1]{\textcolor[rgb]{0.31,0.60,0.02}{#1}}
\newcommand{\StringTok}[1]{\textcolor[rgb]{0.31,0.60,0.02}{#1}}
\newcommand{\VariableTok}[1]{\textcolor[rgb]{0.00,0.00,0.00}{#1}}
\newcommand{\VerbatimStringTok}[1]{\textcolor[rgb]{0.31,0.60,0.02}{#1}}
\newcommand{\WarningTok}[1]{\textcolor[rgb]{0.56,0.35,0.01}{\textbf{\textit{#1}}}}
\usepackage{graphicx}
\makeatletter
\def\maxwidth{\ifdim\Gin@nat@width>\linewidth\linewidth\else\Gin@nat@width\fi}
\def\maxheight{\ifdim\Gin@nat@height>\textheight\textheight\else\Gin@nat@height\fi}
\makeatother
% Scale images if necessary, so that they will not overflow the page
% margins by default, and it is still possible to overwrite the defaults
% using explicit options in \includegraphics[width, height, ...]{}
\setkeys{Gin}{width=\maxwidth,height=\maxheight,keepaspectratio}
% Set default figure placement to htbp
\makeatletter
\def\fps@figure{htbp}
\makeatother
\setlength{\emergencystretch}{3em} % prevent overfull lines
\providecommand{\tightlist}{%
  \setlength{\itemsep}{0pt}\setlength{\parskip}{0pt}}
\setcounter{secnumdepth}{-\maxdimen} % remove section numbering
\ifLuaTeX
  \usepackage{selnolig}  % disable illegal ligatures
\fi
\IfFileExists{bookmark.sty}{\usepackage{bookmark}}{\usepackage{hyperref}}
\IfFileExists{xurl.sty}{\usepackage{xurl}}{} % add URL line breaks if available
\urlstyle{same} % disable monospaced font for URLs
\hypersetup{
  pdftitle={HUDM6026 Homework\_01},
  pdfauthor={Chenguang Pan},
  hidelinks,
  pdfcreator={LaTeX via pandoc}}

\title{HUDM6026 Homework\_01}
\author{Chenguang Pan}
\date{Feb 03, 2023}

\begin{document}
\maketitle

\hypertarget{question-01-scr-3.3}{%
\subsection{Question 01 SCR 3.3}\label{question-01-scr-3.3}}

\textbf{MY SOLUTION:}\\
The inverse transformation of the \texttt{Pareto(a,b)}'s cdf function is
as followed. \[F^{-1}(u)=\frac{b}{(1-u)^\frac{1}{a}} \]

\begin{Shaded}
\begin{Highlighting}[]
\SpecialCharTok{\textgreater{}} \CommentTok{\# define the quantile function of Pareto(a,b) distribution}
\ErrorTok{\textgreater{}}\NormalTok{ quantile\_Pareto }\OtherTok{\textless{}{-}}\ControlFlowTok{function}\NormalTok{(prob, a, b)\{}
\SpecialCharTok{+}\NormalTok{   x }\OtherTok{\textless{}{-}}\NormalTok{ b }\SpecialCharTok{*}\NormalTok{ (}\DecValTok{1}\SpecialCharTok{{-}}\NormalTok{prob)}\SpecialCharTok{\^{}}\NormalTok{(}\SpecialCharTok{{-}}\DecValTok{1}\SpecialCharTok{/}\NormalTok{a)}
\SpecialCharTok{+}   \FunctionTok{return}\NormalTok{(x)}
\SpecialCharTok{+}\NormalTok{ \}}
\SpecialCharTok{\textgreater{}} \CommentTok{\# define the simulated sample size}
\ErrorTok{\textgreater{}}\NormalTok{ n }\OtherTok{\textless{}{-}} \DecValTok{100}
\SpecialCharTok{\textgreater{}}\NormalTok{ u }\OtherTok{\textless{}{-}} \FunctionTok{runif}\NormalTok{(n)}
\SpecialCharTok{\textgreater{}} \CommentTok{\# based on the uniformly generated vector to get the random sample}
\ErrorTok{\textgreater{}}\NormalTok{ X }\OtherTok{\textless{}{-}} \FunctionTok{quantile\_Pareto}\NormalTok{(u, }\DecValTok{2}\NormalTok{, }\DecValTok{2}\NormalTok{)}
\SpecialCharTok{\textgreater{}} \FunctionTok{range}\NormalTok{(X)}
\NormalTok{[}\DecValTok{1}\NormalTok{]  }\FloatTok{2.018862} \FloatTok{12.215290}
\end{Highlighting}
\end{Shaded}

This inverse function runs well. Before comparing the simulated density
and the original density, I derivate the CDF to get the pdf function of
Pareto(a,b), that is:\[f(x)=\frac{ab^a}{x^{a+1}}\]

\begin{Shaded}
\begin{Highlighting}[]
\SpecialCharTok{\textgreater{}} \CommentTok{\# define the layout of graph\textquotesingle{}s output}
\ErrorTok{\textgreater{}} \CommentTok{\#par(mfrow=c(1,2))}
\ErrorTok{\textgreater{}} \CommentTok{\# draw the density histogram of the simulated data}
\ErrorTok{\textgreater{}} \FunctionTok{hist}\NormalTok{(X, }\AttributeTok{prob =}\NormalTok{ T, }
\SpecialCharTok{+}      \AttributeTok{breaks =} \DecValTok{50}\NormalTok{, }
\SpecialCharTok{+}      \AttributeTok{main =} \FunctionTok{expression}\NormalTok{(}\FunctionTok{f}\NormalTok{(x)}\SpecialCharTok{==}\NormalTok{ab}\SpecialCharTok{\^{}}\NormalTok{a}\SpecialCharTok{/}\NormalTok{x}\SpecialCharTok{\^{}}\NormalTok{(a}\SpecialCharTok{+}\DecValTok{1}\NormalTok{))) }
\SpecialCharTok{\textgreater{}} \CommentTok{\# prepare the Pareto(2,2) distribution}
\ErrorTok{\textgreater{}}\NormalTok{ x }\OtherTok{\textless{}{-}} \FunctionTok{seq}\NormalTok{(}\DecValTok{2}\NormalTok{,}\DecValTok{40}\NormalTok{,.}\DecValTok{38}\NormalTok{)}
\SpecialCharTok{\textgreater{}}\NormalTok{ y }\OtherTok{\textless{}{-}} \DecValTok{2}\SpecialCharTok{*}\NormalTok{(}\DecValTok{2}\SpecialCharTok{\^{}}\DecValTok{2}\NormalTok{)}\SpecialCharTok{/}\NormalTok{(x}\SpecialCharTok{\^{}}\NormalTok{(}\DecValTok{2}\SpecialCharTok{+}\DecValTok{1}\NormalTok{))}
\SpecialCharTok{\textgreater{}} \CommentTok{\# superimpose the lines on the simulated density}
\ErrorTok{\textgreater{}} \FunctionTok{lines}\NormalTok{(x, y, }\AttributeTok{col=}\StringTok{"red"}\NormalTok{)}
\SpecialCharTok{\textgreater{}} \FunctionTok{mtext}\NormalTok{(}\StringTok{"Figure 1. Comparing the simulated data with Pareto(a,b)"}\NormalTok{,}
\SpecialCharTok{+}       \AttributeTok{side =} \DecValTok{3}\NormalTok{,}
\SpecialCharTok{+}       \AttributeTok{line =} \SpecialCharTok{{-}}\DecValTok{1}\NormalTok{,}
\SpecialCharTok{+}       \AttributeTok{outer =}\NormalTok{ T)}
\end{Highlighting}
\end{Shaded}

\includegraphics[width=1\linewidth,height=0.5\textheight]{HW_02_Chenguang_Pan_files/figure-latex/unnamed-chunk-2-1}

\end{document}
