% Options for packages loaded elsewhere
\PassOptionsToPackage{unicode}{hyperref}
\PassOptionsToPackage{hyphens}{url}
%
\documentclass[
]{article}
\usepackage{amsmath,amssymb}
\usepackage{lmodern}
\usepackage{iftex}
\ifPDFTeX
  \usepackage[T1]{fontenc}
  \usepackage[utf8]{inputenc}
  \usepackage{textcomp} % provide euro and other symbols
\else % if luatex or xetex
  \usepackage{unicode-math}
  \defaultfontfeatures{Scale=MatchLowercase}
  \defaultfontfeatures[\rmfamily]{Ligatures=TeX,Scale=1}
\fi
% Use upquote if available, for straight quotes in verbatim environments
\IfFileExists{upquote.sty}{\usepackage{upquote}}{}
\IfFileExists{microtype.sty}{% use microtype if available
  \usepackage[]{microtype}
  \UseMicrotypeSet[protrusion]{basicmath} % disable protrusion for tt fonts
}{}
\makeatletter
\@ifundefined{KOMAClassName}{% if non-KOMA class
  \IfFileExists{parskip.sty}{%
    \usepackage{parskip}
  }{% else
    \setlength{\parindent}{0pt}
    \setlength{\parskip}{6pt plus 2pt minus 1pt}}
}{% if KOMA class
  \KOMAoptions{parskip=half}}
\makeatother
\usepackage{xcolor}
\usepackage[margin=1in]{geometry}
\usepackage{color}
\usepackage{fancyvrb}
\newcommand{\VerbBar}{|}
\newcommand{\VERB}{\Verb[commandchars=\\\{\}]}
\DefineVerbatimEnvironment{Highlighting}{Verbatim}{commandchars=\\\{\}}
% Add ',fontsize=\small' for more characters per line
\usepackage{framed}
\definecolor{shadecolor}{RGB}{248,248,248}
\newenvironment{Shaded}{\begin{snugshade}}{\end{snugshade}}
\newcommand{\AlertTok}[1]{\textcolor[rgb]{0.94,0.16,0.16}{#1}}
\newcommand{\AnnotationTok}[1]{\textcolor[rgb]{0.56,0.35,0.01}{\textbf{\textit{#1}}}}
\newcommand{\AttributeTok}[1]{\textcolor[rgb]{0.77,0.63,0.00}{#1}}
\newcommand{\BaseNTok}[1]{\textcolor[rgb]{0.00,0.00,0.81}{#1}}
\newcommand{\BuiltInTok}[1]{#1}
\newcommand{\CharTok}[1]{\textcolor[rgb]{0.31,0.60,0.02}{#1}}
\newcommand{\CommentTok}[1]{\textcolor[rgb]{0.56,0.35,0.01}{\textit{#1}}}
\newcommand{\CommentVarTok}[1]{\textcolor[rgb]{0.56,0.35,0.01}{\textbf{\textit{#1}}}}
\newcommand{\ConstantTok}[1]{\textcolor[rgb]{0.00,0.00,0.00}{#1}}
\newcommand{\ControlFlowTok}[1]{\textcolor[rgb]{0.13,0.29,0.53}{\textbf{#1}}}
\newcommand{\DataTypeTok}[1]{\textcolor[rgb]{0.13,0.29,0.53}{#1}}
\newcommand{\DecValTok}[1]{\textcolor[rgb]{0.00,0.00,0.81}{#1}}
\newcommand{\DocumentationTok}[1]{\textcolor[rgb]{0.56,0.35,0.01}{\textbf{\textit{#1}}}}
\newcommand{\ErrorTok}[1]{\textcolor[rgb]{0.64,0.00,0.00}{\textbf{#1}}}
\newcommand{\ExtensionTok}[1]{#1}
\newcommand{\FloatTok}[1]{\textcolor[rgb]{0.00,0.00,0.81}{#1}}
\newcommand{\FunctionTok}[1]{\textcolor[rgb]{0.00,0.00,0.00}{#1}}
\newcommand{\ImportTok}[1]{#1}
\newcommand{\InformationTok}[1]{\textcolor[rgb]{0.56,0.35,0.01}{\textbf{\textit{#1}}}}
\newcommand{\KeywordTok}[1]{\textcolor[rgb]{0.13,0.29,0.53}{\textbf{#1}}}
\newcommand{\NormalTok}[1]{#1}
\newcommand{\OperatorTok}[1]{\textcolor[rgb]{0.81,0.36,0.00}{\textbf{#1}}}
\newcommand{\OtherTok}[1]{\textcolor[rgb]{0.56,0.35,0.01}{#1}}
\newcommand{\PreprocessorTok}[1]{\textcolor[rgb]{0.56,0.35,0.01}{\textit{#1}}}
\newcommand{\RegionMarkerTok}[1]{#1}
\newcommand{\SpecialCharTok}[1]{\textcolor[rgb]{0.00,0.00,0.00}{#1}}
\newcommand{\SpecialStringTok}[1]{\textcolor[rgb]{0.31,0.60,0.02}{#1}}
\newcommand{\StringTok}[1]{\textcolor[rgb]{0.31,0.60,0.02}{#1}}
\newcommand{\VariableTok}[1]{\textcolor[rgb]{0.00,0.00,0.00}{#1}}
\newcommand{\VerbatimStringTok}[1]{\textcolor[rgb]{0.31,0.60,0.02}{#1}}
\newcommand{\WarningTok}[1]{\textcolor[rgb]{0.56,0.35,0.01}{\textbf{\textit{#1}}}}
\usepackage{graphicx}
\makeatletter
\def\maxwidth{\ifdim\Gin@nat@width>\linewidth\linewidth\else\Gin@nat@width\fi}
\def\maxheight{\ifdim\Gin@nat@height>\textheight\textheight\else\Gin@nat@height\fi}
\makeatother
% Scale images if necessary, so that they will not overflow the page
% margins by default, and it is still possible to overwrite the defaults
% using explicit options in \includegraphics[width, height, ...]{}
\setkeys{Gin}{width=\maxwidth,height=\maxheight,keepaspectratio}
% Set default figure placement to htbp
\makeatletter
\def\fps@figure{htbp}
\makeatother
\setlength{\emergencystretch}{3em} % prevent overfull lines
\providecommand{\tightlist}{%
  \setlength{\itemsep}{0pt}\setlength{\parskip}{0pt}}
\setcounter{secnumdepth}{-\maxdimen} % remove section numbering
\ifLuaTeX
  \usepackage{selnolig}  % disable illegal ligatures
\fi
\IfFileExists{bookmark.sty}{\usepackage{bookmark}}{\usepackage{hyperref}}
\IfFileExists{xurl.sty}{\usepackage{xurl}}{} % add URL line breaks if available
\urlstyle{same} % disable monospaced font for URLs
\hypersetup{
  pdftitle={HUDM6026 Homework\_02},
  pdfauthor={Chenguang Pan},
  hidelinks,
  pdfcreator={LaTeX via pandoc}}

\title{HUDM6026 Homework\_02}
\author{Chenguang Pan}
\date{Feb 03, 2023}

\begin{document}
\maketitle

\hypertarget{question-01-scr-3.3}{%
\subsection{Question 01 SCR 3.3}\label{question-01-scr-3.3}}

\textbf{MY SOLUTION:}\\
The inverse transformation of the \texttt{Pareto(a,b)}'s cdf function is
as followed. \[F^{-1}(u)=\frac{b}{(1-u)^\frac{1}{a}} \]

\begin{Shaded}
\begin{Highlighting}[]
\SpecialCharTok{\textgreater{}} \CommentTok{\# define the quantile function of Pareto(a,b) distribution}
\ErrorTok{\textgreater{}}\NormalTok{ quantile\_Pareto }\OtherTok{\textless{}{-}}\ControlFlowTok{function}\NormalTok{(prob, a, b)\{}
\SpecialCharTok{+}\NormalTok{   x }\OtherTok{\textless{}{-}}\NormalTok{ b }\SpecialCharTok{*}\NormalTok{ (}\DecValTok{1}\SpecialCharTok{{-}}\NormalTok{prob)}\SpecialCharTok{\^{}}\NormalTok{(}\SpecialCharTok{{-}}\DecValTok{1}\SpecialCharTok{/}\NormalTok{a)}
\SpecialCharTok{+}   \FunctionTok{return}\NormalTok{(x)}
\SpecialCharTok{+}\NormalTok{ \}}
\SpecialCharTok{\textgreater{}} \CommentTok{\# define the simulated sample size}
\ErrorTok{\textgreater{}}\NormalTok{ n }\OtherTok{\textless{}{-}} \DecValTok{100}
\SpecialCharTok{\textgreater{}}\NormalTok{ u }\OtherTok{\textless{}{-}} \FunctionTok{runif}\NormalTok{(n)}
\SpecialCharTok{\textgreater{}} \CommentTok{\# based on the uniformly generated vector to get the random sample}
\ErrorTok{\textgreater{}}\NormalTok{ X }\OtherTok{\textless{}{-}} \FunctionTok{quantile\_Pareto}\NormalTok{(u, }\DecValTok{2}\NormalTok{, }\DecValTok{2}\NormalTok{)}
\SpecialCharTok{\textgreater{}} \FunctionTok{range}\NormalTok{(X)}
\NormalTok{[}\DecValTok{1}\NormalTok{]  }\FloatTok{2.018862} \FloatTok{12.215290}
\end{Highlighting}
\end{Shaded}

This inverse function runs well. Before comparing the simulated density
and the original density, I derivate the CDF to get the pdf function of
Pareto(a,b), that is:\[f(x)=\frac{ab^a}{x^{a+1}}\]

\begin{Shaded}
\begin{Highlighting}[]
\SpecialCharTok{\textgreater{}} \CommentTok{\# draw the density histogram of the simulated data}
\ErrorTok{\textgreater{}} \FunctionTok{hist}\NormalTok{(X, }\AttributeTok{prob =}\NormalTok{ T, }
\SpecialCharTok{+}      \AttributeTok{breaks =} \DecValTok{50}\NormalTok{, }
\SpecialCharTok{+}      \AttributeTok{main =} \FunctionTok{expression}\NormalTok{(}\FunctionTok{f}\NormalTok{(x)}\SpecialCharTok{==}\NormalTok{ab}\SpecialCharTok{\^{}}\NormalTok{a}\SpecialCharTok{/}\NormalTok{x}\SpecialCharTok{\^{}}\NormalTok{(a}\SpecialCharTok{+}\DecValTok{1}\NormalTok{))) }
\SpecialCharTok{\textgreater{}} \CommentTok{\# prepare the Pareto(2,2) distribution}
\ErrorTok{\textgreater{}}\NormalTok{ x }\OtherTok{\textless{}{-}} \FunctionTok{seq}\NormalTok{(}\DecValTok{2}\NormalTok{,}\DecValTok{40}\NormalTok{,.}\DecValTok{38}\NormalTok{)}
\SpecialCharTok{\textgreater{}}\NormalTok{ y }\OtherTok{\textless{}{-}} \DecValTok{2}\SpecialCharTok{*}\NormalTok{(}\DecValTok{2}\SpecialCharTok{\^{}}\DecValTok{2}\NormalTok{)}\SpecialCharTok{/}\NormalTok{(x}\SpecialCharTok{\^{}}\NormalTok{(}\DecValTok{2}\SpecialCharTok{+}\DecValTok{1}\NormalTok{))}
\SpecialCharTok{\textgreater{}} \CommentTok{\# superimpose the lines on the simulated density}
\ErrorTok{\textgreater{}} \FunctionTok{lines}\NormalTok{(x, y, }\AttributeTok{col=}\StringTok{"red"}\NormalTok{)}
\SpecialCharTok{\textgreater{}} \FunctionTok{mtext}\NormalTok{(}\StringTok{"Figure 1. Comparing the simulated data with Pareto(a,b)"}\NormalTok{,}
\SpecialCharTok{+}       \AttributeTok{side =} \DecValTok{3}\NormalTok{,}
\SpecialCharTok{+}       \AttributeTok{line =} \SpecialCharTok{{-}}\DecValTok{1}\NormalTok{,}
\SpecialCharTok{+}       \AttributeTok{outer =}\NormalTok{ T)}
\end{Highlighting}
\end{Shaded}

\includegraphics[width=1\linewidth,height=0.4\textheight]{HW_02_Chenguang_Pan_files/figure-latex/unnamed-chunk-2-1}

\hypertarget{question-02-scr-3.9}{%
\subsection{Question 02 SCR 3.9}\label{question-02-scr-3.9}}

\textbf{MY SOLUTION:}\\
This question has already given the clues to generate random variable
for the rescaled Epanechnikov kernel

\begin{Shaded}
\begin{Highlighting}[]
\SpecialCharTok{\textgreater{}} \CommentTok{\# write a function based on text\textquotesingle{}s information}
\ErrorTok{\textgreater{}}\NormalTok{ gen\_var }\OtherTok{\textless{}{-}} \ControlFlowTok{function}\NormalTok{(n)\{ }\CommentTok{\# n is the sample size}
\SpecialCharTok{+}\NormalTok{   U\_1 }\OtherTok{\textless{}{-}} \FunctionTok{runif}\NormalTok{(n, }\SpecialCharTok{{-}}\DecValTok{1}\NormalTok{, }\DecValTok{1}\NormalTok{)}
\SpecialCharTok{+}\NormalTok{   U\_2 }\OtherTok{\textless{}{-}} \FunctionTok{runif}\NormalTok{(n, }\SpecialCharTok{{-}}\DecValTok{1}\NormalTok{, }\DecValTok{1}\NormalTok{)}
\SpecialCharTok{+}\NormalTok{   U\_3 }\OtherTok{\textless{}{-}} \FunctionTok{runif}\NormalTok{(n, }\SpecialCharTok{{-}}\DecValTok{1}\NormalTok{, }\DecValTok{1}\NormalTok{)}
\SpecialCharTok{+}\NormalTok{   U\_output }\OtherTok{\textless{}{-}} \FunctionTok{c}\NormalTok{()}
\SpecialCharTok{+}   \ControlFlowTok{for}\NormalTok{ (i }\ControlFlowTok{in} \FunctionTok{c}\NormalTok{(}\DecValTok{1}\SpecialCharTok{:}\NormalTok{n)) \{}
\SpecialCharTok{+}     \ControlFlowTok{if}\NormalTok{ (}\FunctionTok{abs}\NormalTok{(U\_3[i]) }\SpecialCharTok{\textgreater{}} \FunctionTok{abs}\NormalTok{(U\_2[i]) }\SpecialCharTok{\&} 
\SpecialCharTok{+}         \FunctionTok{abs}\NormalTok{(U\_3[i]) }\SpecialCharTok{\textgreater{}} \FunctionTok{abs}\NormalTok{(U\_1[i]))}
\SpecialCharTok{+}\NormalTok{       \{U\_output[i] }\OtherTok{\textless{}{-}}\NormalTok{ U\_2[i]\} }
\SpecialCharTok{+}     \ControlFlowTok{else} 
\SpecialCharTok{+}\NormalTok{       \{U\_output[i] }\OtherTok{\textless{}{-}}\NormalTok{ U\_3[i]\}}
\SpecialCharTok{+}\NormalTok{   \}}
\SpecialCharTok{+}   \FunctionTok{return}\NormalTok{(U\_output)}
\SpecialCharTok{+}\NormalTok{ \}}
\SpecialCharTok{\textgreater{}} 
\ErrorTok{\textgreater{}} \CommentTok{\# generate 1000 data}
\ErrorTok{\textgreater{}}\NormalTok{ U\_output }\OtherTok{\textless{}{-}} \FunctionTok{gen\_var}\NormalTok{(}\DecValTok{1000}\NormalTok{)}
\SpecialCharTok{\textgreater{}} \FunctionTok{hist}\NormalTok{(U\_output, }\AttributeTok{prob =}\NormalTok{ T, }
\SpecialCharTok{+}      \AttributeTok{breaks =} \DecValTok{100}\NormalTok{, }
\SpecialCharTok{+}      \AttributeTok{xlab =} \StringTok{"x"}\NormalTok{,}
\SpecialCharTok{+}      \AttributeTok{main =} \FunctionTok{expression}\NormalTok{(}\FunctionTok{f}\NormalTok{(x)}\SpecialCharTok{==}\NormalTok{(}\DecValTok{3}\SpecialCharTok{/}\DecValTok{4}\NormalTok{)}\SpecialCharTok{*}\NormalTok{(}\DecValTok{1}\SpecialCharTok{{-}}\NormalTok{x}\SpecialCharTok{\^{}}\DecValTok{2}\NormalTok{)))}
\SpecialCharTok{\textgreater{}}\NormalTok{ x\_vec }\OtherTok{\textless{}{-}} \FunctionTok{seq}\NormalTok{(}\SpecialCharTok{{-}}\DecValTok{1}\NormalTok{,}\DecValTok{1}\NormalTok{,}\FloatTok{0.001}\NormalTok{)}
\SpecialCharTok{\textgreater{}}\NormalTok{ f\_x }\OtherTok{\textless{}{-}} \FloatTok{0.75}\SpecialCharTok{*}\NormalTok{(}\DecValTok{1}\SpecialCharTok{{-}}\NormalTok{x\_vec}\SpecialCharTok{\^{}}\DecValTok{2}\NormalTok{)}
\SpecialCharTok{\textgreater{}} \FunctionTok{lines}\NormalTok{(x\_vec, f\_x, }\AttributeTok{col=}\StringTok{"red"}\NormalTok{)}
\SpecialCharTok{\textgreater{}} \FunctionTok{mtext}\NormalTok{(}\StringTok{"Figure 2. Rescaled Epanechnikov kernel Distribution"}\NormalTok{,}
\SpecialCharTok{+}       \AttributeTok{side =} \DecValTok{3}\NormalTok{,}
\SpecialCharTok{+}       \AttributeTok{line =} \SpecialCharTok{{-}}\DecValTok{1}\NormalTok{,}
\SpecialCharTok{+}       \AttributeTok{outer =}\NormalTok{ T)}
\end{Highlighting}
\end{Shaded}

\includegraphics[width=1\linewidth,height=0.4\textheight]{HW_02_Chenguang_Pan_files/figure-latex/unnamed-chunk-3-1}

\hypertarget{question-03-scr-3.11}{%
\subsection{Question 03 SCR 3.11}\label{question-03-scr-3.11}}

\textbf{MY SOLUTION:}\\
How to better understand the mixing weights (i.e., the mixing
probabilities)? The mixing weights is about \textbf{how much} each
individual distribution contributes to the mixture
distribution(\href{https://www.statisticshowto.com/mixture-distribution/}{Stephanie
Glen.n.d.}). Therefore, when constructing the mixture function, one
should not directly use the probability of each parent distribution as a
coefficient!!

\begin{Shaded}
\begin{Highlighting}[]
\SpecialCharTok{\textgreater{}} \FunctionTok{set.seed}\NormalTok{(}\DecValTok{1000}\NormalTok{)}
\SpecialCharTok{\textgreater{}}\NormalTok{ n }\OtherTok{\textless{}{-}} \DecValTok{1000}
\SpecialCharTok{\textgreater{}} \CommentTok{\# generate two vectors from normal distribution}
\ErrorTok{\textgreater{}}\NormalTok{ x1 }\OtherTok{\textless{}{-}} \FunctionTok{rnorm}\NormalTok{(n,}\DecValTok{0}\NormalTok{,}\DecValTok{1}\NormalTok{)}
\SpecialCharTok{\textgreater{}}\NormalTok{ x2 }\OtherTok{\textless{}{-}} \FunctionTok{rnorm}\NormalTok{(n,}\DecValTok{3}\NormalTok{,}\DecValTok{1}\NormalTok{)}
\SpecialCharTok{\textgreater{}} 
\ErrorTok{\textgreater{}} \CommentTok{\# use a for{-}loop to draw graphs at different mixing weights}
\ErrorTok{\textgreater{}} \FunctionTok{par}\NormalTok{(}\AttributeTok{mfrow=}\FunctionTok{c}\NormalTok{(}\DecValTok{3}\NormalTok{,}\DecValTok{2}\NormalTok{))}
\SpecialCharTok{\textgreater{}} \ControlFlowTok{for}\NormalTok{ (p1 }\ControlFlowTok{in} \FunctionTok{c}\NormalTok{(}\FloatTok{0.75}\NormalTok{, }\FloatTok{0.90}\NormalTok{, }\FloatTok{0.60}\NormalTok{, }\FloatTok{0.5}\NormalTok{, }\FloatTok{0.30}\NormalTok{, }\FloatTok{0.15}\NormalTok{))\{}
\SpecialCharTok{+}   \CommentTok{\# define the mixing prob}
\SpecialCharTok{+}\NormalTok{   p2 }\OtherTok{\textless{}{-}} \DecValTok{1} \SpecialCharTok{{-}}\NormalTok{ p1}
\SpecialCharTok{+}   \CommentTok{\# use n data from uniform distribution to construct}
\SpecialCharTok{+}   \CommentTok{\# the proportion of each parent distribution.}
\SpecialCharTok{+}\NormalTok{   u }\OtherTok{\textless{}{-}} \FunctionTok{runif}\NormalTok{(n)}
\SpecialCharTok{+}\NormalTok{   k }\OtherTok{\textless{}{-}} \FunctionTok{as.integer}\NormalTok{(u }\SpecialCharTok{\textgreater{}}\NormalTok{ p2)}
\SpecialCharTok{+}\NormalTok{   x }\OtherTok{\textless{}{-}}\NormalTok{ k }\SpecialCharTok{*}\NormalTok{ x1 }\SpecialCharTok{+}\NormalTok{ (}\DecValTok{1}\SpecialCharTok{{-}}\NormalTok{k) }\SpecialCharTok{*}\NormalTok{ x2}
\SpecialCharTok{+}   \FunctionTok{hist}\NormalTok{(x, }\AttributeTok{prob =}\NormalTok{ T, }
\SpecialCharTok{+}        \AttributeTok{breaks =} \DecValTok{50}\NormalTok{, }
\SpecialCharTok{+}        \AttributeTok{xlab =} \StringTok{"mixture x"}\NormalTok{,}
\SpecialCharTok{+}        \AttributeTok{main =} \FunctionTok{sprintf}\NormalTok{(}\StringTok{"p1=\%s, p2=\%s"}\NormalTok{, p1, p2))}
\SpecialCharTok{+}   \CommentTok{\# using weighted sum of dnorm() to construct true density function}
\SpecialCharTok{+}\NormalTok{   x\_true }\OtherTok{\textless{}{-}} \FunctionTok{seq}\NormalTok{(}\SpecialCharTok{{-}}\DecValTok{10}\NormalTok{,}\DecValTok{10}\NormalTok{,}\FloatTok{0.1}\NormalTok{)}
\SpecialCharTok{+}\NormalTok{   y\_true }\OtherTok{\textless{}{-}}\NormalTok{ p1}\SpecialCharTok{*}\FunctionTok{dnorm}\NormalTok{(x\_true,}\DecValTok{0}\NormalTok{,}\DecValTok{1}\NormalTok{) }\SpecialCharTok{+}\NormalTok{ (}\DecValTok{1}\SpecialCharTok{{-}}\NormalTok{p1)}\SpecialCharTok{*}\FunctionTok{dnorm}\NormalTok{(x\_true,}\DecValTok{3}\NormalTok{,}\DecValTok{1}\NormalTok{)}
\SpecialCharTok{+}   \FunctionTok{lines}\NormalTok{(x\_true,y\_true,}\AttributeTok{col=}\StringTok{"red"}\NormalTok{)}
\SpecialCharTok{+}\NormalTok{ \}}
\SpecialCharTok{\textgreater{}} \FunctionTok{mtext}\NormalTok{(}\StringTok{"Figure 3. Mixture Distributions With Different Mixing Weights"}\NormalTok{,}
\SpecialCharTok{+}       \AttributeTok{side =} \DecValTok{3}\NormalTok{,}
\SpecialCharTok{+}       \AttributeTok{line =} \SpecialCharTok{{-}}\DecValTok{1}\NormalTok{,}
\SpecialCharTok{+}       \AttributeTok{outer =}\NormalTok{ T)}
\end{Highlighting}
\end{Shaded}

\includegraphics{HW_02_Chenguang_Pan_files/figure-latex/unnamed-chunk-4-1.pdf}
From the graphs, one can find that when the mixing weights are .5 and
.5, the mixture distribution is apparently a bimodal distribution. At
this circumstance the two samples contribute equally to the final
mixture. But this bimodal distribution might not be symmetrical because
the two parent distribution's shapes are different in variance. With
increasing in P1, the bimodal distribution will be more positively
skewed.

\hypertarget{question-04-scr-3.14}{%
\subsection{Question 04 SCR 3.14}\label{question-04-scr-3.14}}

\textbf{MY SOLUTION:}\\
For solving this question, I refer to Prof.Keller's in-class demo codes
and remove unnecessary if-else expressions.

\begin{Shaded}
\begin{Highlighting}[]
\SpecialCharTok{\textgreater{}} \CommentTok{\# create a function to gen data matrix from a multivariate normal distribution.}
\ErrorTok{\textgreater{}}\NormalTok{ mvn\_gen }\OtherTok{\textless{}{-}} \ControlFlowTok{function}\NormalTok{(n, mu, sigma)\{}
\SpecialCharTok{+}   \CommentTok{\# to determine the dimension}
\SpecialCharTok{+}\NormalTok{   d }\OtherTok{\textless{}{-}} \FunctionTok{length}\NormalTok{(mu)}
\SpecialCharTok{+}   \CommentTok{\# generate a n*d matrix from a standard normal distribution}
\SpecialCharTok{+}\NormalTok{   Z }\OtherTok{\textless{}{-}} \FunctionTok{matrix}\NormalTok{(}\FunctionTok{rnorm}\NormalTok{(n}\SpecialCharTok{*}\NormalTok{d), }\AttributeTok{nrow =}\NormalTok{ n, }\AttributeTok{ncol =}\NormalTok{ d)}
\SpecialCharTok{+}   \CommentTok{\# use Cholesky function to factorize the Cov{-}var matrix}
\SpecialCharTok{+}\NormalTok{   Q }\OtherTok{\textless{}{-}} \FunctionTok{chol}\NormalTok{(sigma)}
\SpecialCharTok{+}   \CommentTok{\# to transform the mu from dataframe to matrix}
\SpecialCharTok{+}\NormalTok{   mu }\OtherTok{\textless{}{-}} \FunctionTok{matrix}\NormalTok{ (mu, }\AttributeTok{nrow =}\NormalTok{ d, }\AttributeTok{ncol =} \DecValTok{1}\NormalTok{)}
\SpecialCharTok{+}   \CommentTok{\# J is column vector of ones}
\SpecialCharTok{+}\NormalTok{   J }\OtherTok{\textless{}{-}} \FunctionTok{matrix}\NormalTok{(mu, }\AttributeTok{nrow =}\NormalTok{ n, }\AttributeTok{ncol =} \DecValTok{1}\NormalTok{)}
\SpecialCharTok{+}   \CommentTok{\# define the output X matrix}
\SpecialCharTok{+}\NormalTok{   X }\OtherTok{\textless{}{-}}\NormalTok{ Z }\SpecialCharTok{\%*\%}\NormalTok{ Q }\SpecialCharTok{+}\NormalTok{ J }\SpecialCharTok{\%*\%} \FunctionTok{t}\NormalTok{(mu)}
\SpecialCharTok{+}   \FunctionTok{return}\NormalTok{(}\FunctionTok{data.frame}\NormalTok{(X))}
\SpecialCharTok{+}\NormalTok{ \}}
\SpecialCharTok{\textgreater{}} 
\ErrorTok{\textgreater{}} \CommentTok{\# input the given cov{-}var matrix}
\ErrorTok{\textgreater{}}\NormalTok{ sig }\OtherTok{\textless{}{-}} \FunctionTok{matrix}\NormalTok{(}\FunctionTok{c}\NormalTok{(}\FloatTok{1.0}\NormalTok{, }\SpecialCharTok{{-}}\FloatTok{0.5}\NormalTok{, }\FloatTok{0.5}\NormalTok{,}
\SpecialCharTok{+}                 \SpecialCharTok{{-}}\FloatTok{0.5}\NormalTok{, }\FloatTok{1.0}\NormalTok{, }\SpecialCharTok{{-}}\FloatTok{0.5}\NormalTok{,}
\SpecialCharTok{+}                 \FloatTok{0.5}\NormalTok{, }\SpecialCharTok{{-}}\FloatTok{0.5}\NormalTok{, }\FloatTok{1.0}\NormalTok{),}\DecValTok{3}\NormalTok{,}\DecValTok{3}\NormalTok{)}
\SpecialCharTok{\textgreater{}} \FunctionTok{set.seed}\NormalTok{(}\DecValTok{100}\NormalTok{)}
\SpecialCharTok{\textgreater{}}\NormalTok{ X }\OtherTok{\textless{}{-}} \FunctionTok{mvn\_gen}\NormalTok{(}\AttributeTok{n=}\DecValTok{200}\NormalTok{, }\AttributeTok{mu =} \FunctionTok{c}\NormalTok{(}\DecValTok{0}\NormalTok{,}\DecValTok{1}\NormalTok{,}\DecValTok{2}\NormalTok{), }\AttributeTok{sigma =}\NormalTok{ sig)}
\SpecialCharTok{\textgreater{}} \FunctionTok{pairs}\NormalTok{(X)}
\SpecialCharTok{\textgreater{}} \FunctionTok{mtext}\NormalTok{(}\StringTok{"Figure 4. Generate Data From Multivariate Normal Distribution"}\NormalTok{,}
\SpecialCharTok{+}       \AttributeTok{side =} \DecValTok{3}\NormalTok{,}
\SpecialCharTok{+}       \AttributeTok{line =} \SpecialCharTok{{-}}\DecValTok{1}\NormalTok{,}
\SpecialCharTok{+}       \AttributeTok{outer =}\NormalTok{ T)}
\end{Highlighting}
\end{Shaded}

\includegraphics[width=1\linewidth,height=0.5\textheight]{HW_02_Chenguang_Pan_files/figure-latex/unnamed-chunk-5-1}

From the cov-var matrix, one can easily get the correlation coefficient
between the \(x_1\) and \(x_2\) is -0.5. The middle-right graph
demonstrated that these two variables are negatively correlated. By
visually check, the joint mean point is at around\texttt{(0,1)}, which
agrees with the given condition. The bottom-right graph also meets the
given condition. However, the correlation between the \(x_2\) and
\(x_3\) is obscure by visual check.

\hypertarget{question-05-bonous}{%
\subsection{Question 05 {[}Bonous{]}}\label{question-05-bonous}}

\emph{First show that the sample mean estimator is unbiased for the true
population mean. Next, show that the mle estimator for the
variance\ldots{}}

\textbf{MY SOLUTION for Part 1:}

Before proving the first statement, I want to refresh the understanding
of several stats concepts. By definition, an \textbf{estimator} is a
rule that is used to estimate an unknown parameter based on sample. The
concept \textbf{Unbiased} means that the expectation of an estimator
equals to the population parameter, i.e.,
\(E(estimator) = Population Paramter\).

The sample mean estimator is
\[\overline{x} = \frac{1}{n}\sum_{i = 1}^{n} x_{i}\], where
\(\overline{x}\) is the mean of a sample with n size. therefore,
\[E(estimator) = E(\overline{x})=E(\frac{1}{n}\sum_{i = 1}^{n} x_{i})= \frac{1}{n}\sum_{i = 1}^{n}E(x_i)\].
Since \(x_i\) is draw from the population with mean \(\mu\), one can
have \(E(x_i) = \mu\). Put this equation to the above, I get
\[E(estimator) = E(\overline{x}) = \frac{1}{n}\sum_{i = 1}^{n}\mu=\frac{1}{n}n\mu=\mu\].
In short, here I have \[E(\overline{x})= \mu\], which proved the sample
mean estimator is unbiased for the true population mean.

\textbf{MY SOLUTION for Part 2:}\\
By definition, the maximum likelihood estimator for the variance is
\[s^2=\frac{1}{n}\sum_{i = 1}^{n}(x_i-\overline{x})^2\]. The variance of
the population is \[\sigma^2=\frac{1}{N}\sum_{i = 1}^{N}(x_i-{\mu})^2\].
Visually checking the two equations, it seems identical. Based on the
definition of ``unbiased'', we need to further check whether the two
above follow this rule \(E(estimator) = Population Paramter\).

Here, I have
\[E(estimator) = E(s^2)=E(\frac{1}{n}\sum_{i = 1}^{n}(x_i-\overline{x})^2)=\\\frac{1}{n}E(\sum_{i = 1}^{n}{x_i}^2 - 2\sum_{i = 1}^{n}x_i\overline{x}+\sum_{i = 1}^{n}\overline{x}^2)\].
Because \(\sum_{i = 1}^{n}x_i = n\overline{x}\), combining these two
equation above one can have
\[E(estimator)=E(s^2)=\frac{1}{n}E(\sum_{i = 1}^{n}{x_i}^2 - 2n\overline{x}\overline{x}+n\overline{x}^2)=\\E(\frac{1}{n}\sum_{i = 1}^{n}{x_i}^2)-E(\overline{x}^2)=\\E(x^2)-E(\overline{x}^2)\].

Based on the definition of variance, we can have a equation of
population variance, that is,
\[\sigma^2 = E\{[x - E(x)]^2\}=E[x^2-2xE(x)+E(x)^2]=E(x^2)-E(x)^2\].
Based on this formula, we can also have
\[\sigma_{\overline{x}}^2=E(\overline{x}^2)-E(\overline{x})^2\]. Since
\(E(x)=E(\overline{x})=\mu\), based on all the equation above, one can
have \[E(estimator)=E(s^2)= {\sigma}^2 - \sigma_{\overline{x}}^2\]. We
need to go further to derive the \(\sigma_{\overline{x}}^2\). Here, by
the rule of variance, one can have
\[\sigma_{\overline{x}}^2 = var(\overline{x})=var(\frac{1}{n}\sum_{i = 1}^{n}x_i)=\frac{1}{n^2}var(\sum_{i = 1}^{n}x_i)\].
Because the samples are independently and identically drawn from the
population, the covariance among them should be zero, we can have
\[\sigma_{\overline{x}}^2 =\frac{1}{n^2}\sum_{i = 1}^{n}var(x)=\frac{1}{n}var(x)=\frac{1}{n}{\sigma}^2\].From
the equations above, one can find the \emph{mle} estimator of variance
does not follow the rule \(E(estimator) = Population Paramter\), which
means this estimator is biased. The difference between this estimator
and the true population variance is \(\frac{1}{n}{\sigma}^2\), which
represents the bias.
\[E(s^2)-{\sigma}^2 = -\sigma_{\overline{x}}^2=-\frac{1}{n}{\sigma}^2\]

\hypertarget{reference-for-this-homework}{%
\subsection{Reference For this
Homework}\label{reference-for-this-homework}}

Glen,S.(n.d.). \emph{Mixture Distribution: Definition and Examples.}
\url{https://www.statisticshowto.com/mixture-distribution/}

Keller, B.(2023). \emph{HUDM 6026 Computational Statistics: Simulation
and Other Monte Carlo Methods}{[}Lecture notes{]}.

Liang, D. (2012). \emph{Maximum likelihood estimator for variance is
biased: Proof.} \url{https://dawenl.github.io/files/mle_biased.pdf}

Rizzo, M. L. (2019). \emph{Statistical computing with R.} Chapman and
Hall/CRC.

\end{document}
