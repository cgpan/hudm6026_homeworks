% Options for packages loaded elsewhere
\PassOptionsToPackage{unicode}{hyperref}
\PassOptionsToPackage{hyphens}{url}
%
\documentclass[
]{article}
\usepackage{amsmath,amssymb}
\usepackage{lmodern}
\usepackage{iftex}
\ifPDFTeX
  \usepackage[T1]{fontenc}
  \usepackage[utf8]{inputenc}
  \usepackage{textcomp} % provide euro and other symbols
\else % if luatex or xetex
  \usepackage{unicode-math}
  \defaultfontfeatures{Scale=MatchLowercase}
  \defaultfontfeatures[\rmfamily]{Ligatures=TeX,Scale=1}
\fi
% Use upquote if available, for straight quotes in verbatim environments
\IfFileExists{upquote.sty}{\usepackage{upquote}}{}
\IfFileExists{microtype.sty}{% use microtype if available
  \usepackage[]{microtype}
  \UseMicrotypeSet[protrusion]{basicmath} % disable protrusion for tt fonts
}{}
\makeatletter
\@ifundefined{KOMAClassName}{% if non-KOMA class
  \IfFileExists{parskip.sty}{%
    \usepackage{parskip}
  }{% else
    \setlength{\parindent}{0pt}
    \setlength{\parskip}{6pt plus 2pt minus 1pt}}
}{% if KOMA class
  \KOMAoptions{parskip=half}}
\makeatother
\usepackage{xcolor}
\usepackage[margin=1in]{geometry}
\usepackage{color}
\usepackage{fancyvrb}
\newcommand{\VerbBar}{|}
\newcommand{\VERB}{\Verb[commandchars=\\\{\}]}
\DefineVerbatimEnvironment{Highlighting}{Verbatim}{commandchars=\\\{\}}
% Add ',fontsize=\small' for more characters per line
\usepackage{framed}
\definecolor{shadecolor}{RGB}{248,248,248}
\newenvironment{Shaded}{\begin{snugshade}}{\end{snugshade}}
\newcommand{\AlertTok}[1]{\textcolor[rgb]{0.94,0.16,0.16}{#1}}
\newcommand{\AnnotationTok}[1]{\textcolor[rgb]{0.56,0.35,0.01}{\textbf{\textit{#1}}}}
\newcommand{\AttributeTok}[1]{\textcolor[rgb]{0.77,0.63,0.00}{#1}}
\newcommand{\BaseNTok}[1]{\textcolor[rgb]{0.00,0.00,0.81}{#1}}
\newcommand{\BuiltInTok}[1]{#1}
\newcommand{\CharTok}[1]{\textcolor[rgb]{0.31,0.60,0.02}{#1}}
\newcommand{\CommentTok}[1]{\textcolor[rgb]{0.56,0.35,0.01}{\textit{#1}}}
\newcommand{\CommentVarTok}[1]{\textcolor[rgb]{0.56,0.35,0.01}{\textbf{\textit{#1}}}}
\newcommand{\ConstantTok}[1]{\textcolor[rgb]{0.00,0.00,0.00}{#1}}
\newcommand{\ControlFlowTok}[1]{\textcolor[rgb]{0.13,0.29,0.53}{\textbf{#1}}}
\newcommand{\DataTypeTok}[1]{\textcolor[rgb]{0.13,0.29,0.53}{#1}}
\newcommand{\DecValTok}[1]{\textcolor[rgb]{0.00,0.00,0.81}{#1}}
\newcommand{\DocumentationTok}[1]{\textcolor[rgb]{0.56,0.35,0.01}{\textbf{\textit{#1}}}}
\newcommand{\ErrorTok}[1]{\textcolor[rgb]{0.64,0.00,0.00}{\textbf{#1}}}
\newcommand{\ExtensionTok}[1]{#1}
\newcommand{\FloatTok}[1]{\textcolor[rgb]{0.00,0.00,0.81}{#1}}
\newcommand{\FunctionTok}[1]{\textcolor[rgb]{0.00,0.00,0.00}{#1}}
\newcommand{\ImportTok}[1]{#1}
\newcommand{\InformationTok}[1]{\textcolor[rgb]{0.56,0.35,0.01}{\textbf{\textit{#1}}}}
\newcommand{\KeywordTok}[1]{\textcolor[rgb]{0.13,0.29,0.53}{\textbf{#1}}}
\newcommand{\NormalTok}[1]{#1}
\newcommand{\OperatorTok}[1]{\textcolor[rgb]{0.81,0.36,0.00}{\textbf{#1}}}
\newcommand{\OtherTok}[1]{\textcolor[rgb]{0.56,0.35,0.01}{#1}}
\newcommand{\PreprocessorTok}[1]{\textcolor[rgb]{0.56,0.35,0.01}{\textit{#1}}}
\newcommand{\RegionMarkerTok}[1]{#1}
\newcommand{\SpecialCharTok}[1]{\textcolor[rgb]{0.00,0.00,0.00}{#1}}
\newcommand{\SpecialStringTok}[1]{\textcolor[rgb]{0.31,0.60,0.02}{#1}}
\newcommand{\StringTok}[1]{\textcolor[rgb]{0.31,0.60,0.02}{#1}}
\newcommand{\VariableTok}[1]{\textcolor[rgb]{0.00,0.00,0.00}{#1}}
\newcommand{\VerbatimStringTok}[1]{\textcolor[rgb]{0.31,0.60,0.02}{#1}}
\newcommand{\WarningTok}[1]{\textcolor[rgb]{0.56,0.35,0.01}{\textbf{\textit{#1}}}}
\usepackage{graphicx}
\makeatletter
\def\maxwidth{\ifdim\Gin@nat@width>\linewidth\linewidth\else\Gin@nat@width\fi}
\def\maxheight{\ifdim\Gin@nat@height>\textheight\textheight\else\Gin@nat@height\fi}
\makeatother
% Scale images if necessary, so that they will not overflow the page
% margins by default, and it is still possible to overwrite the defaults
% using explicit options in \includegraphics[width, height, ...]{}
\setkeys{Gin}{width=\maxwidth,height=\maxheight,keepaspectratio}
% Set default figure placement to htbp
\makeatletter
\def\fps@figure{htbp}
\makeatother
\setlength{\emergencystretch}{3em} % prevent overfull lines
\providecommand{\tightlist}{%
  \setlength{\itemsep}{0pt}\setlength{\parskip}{0pt}}
\setcounter{secnumdepth}{-\maxdimen} % remove section numbering
\ifLuaTeX
  \usepackage{selnolig}  % disable illegal ligatures
\fi
\IfFileExists{bookmark.sty}{\usepackage{bookmark}}{\usepackage{hyperref}}
\IfFileExists{xurl.sty}{\usepackage{xurl}}{} % add URL line breaks if available
\urlstyle{same} % disable monospaced font for URLs
\hypersetup{
  pdftitle={HUDM6026 Homework\_05},
  pdfauthor={Chenguang Pan \& Seng Lei},
  hidelinks,
  pdfcreator={LaTeX via pandoc}}

\title{HUDM6026 Homework\_05}
\author{Chenguang Pan \& Seng Lei}
\date{Feb 24, 2023}

\begin{document}
\maketitle

\hypertarget{q1}{%
\subsection{Q1:}\label{q1}}

\emph{Determine the first derivative of f and encode it in a function
called f\_prime.}

\textbf{MY SOLUTION:}\\
Based on chain rule, the first derivative of \(f(x)\) is
\[f(x)' = \frac{-2x}{x^2+1}+\frac{1}{3}x^{-\frac{2}{3}}\]. Based on this
equation, I write the code below. Certainly, we can use the R-built-in
function to get the derivative quickly.

\begin{Shaded}
\begin{Highlighting}[]
\SpecialCharTok{\textgreater{}}\NormalTok{ f\_prime }\OtherTok{\textless{}{-}} \ControlFlowTok{function}\NormalTok{(x) \{}
\SpecialCharTok{+}\NormalTok{   out\_ }\OtherTok{\textless{}{-}}\NormalTok{ (}\SpecialCharTok{{-}}\DecValTok{2}\SpecialCharTok{*}\NormalTok{x)}\SpecialCharTok{/}\NormalTok{(x}\SpecialCharTok{\^{}}\DecValTok{2} \SpecialCharTok{+} \DecValTok{1}\NormalTok{) }\SpecialCharTok{+}\NormalTok{ (}\DecValTok{1}\SpecialCharTok{/}\DecValTok{3}\NormalTok{)}\SpecialCharTok{*}\NormalTok{(x}\SpecialCharTok{\^{}}\NormalTok{(}\SpecialCharTok{{-}}\DecValTok{2}\SpecialCharTok{/}\DecValTok{3}\NormalTok{))}
\SpecialCharTok{+}   \FunctionTok{return}\NormalTok{(out\_)}
\SpecialCharTok{+}\NormalTok{ \}}
\SpecialCharTok{\textgreater{}} \CommentTok{\# since this question requires the maximum, I use {-}1 times the function.}
\ErrorTok{\textgreater{}}\NormalTok{ f\_prime\_neg }\OtherTok{\textless{}{-}} \ControlFlowTok{function}\NormalTok{(x)\{}
\SpecialCharTok{+}\NormalTok{   out\_ }\OtherTok{\textless{}{-}}\NormalTok{ (}\SpecialCharTok{{-}}\DecValTok{2}\SpecialCharTok{*}\NormalTok{x)}\SpecialCharTok{/}\NormalTok{(x}\SpecialCharTok{\^{}}\DecValTok{2} \SpecialCharTok{+} \DecValTok{1}\NormalTok{) }\SpecialCharTok{+}\NormalTok{ (}\DecValTok{1}\SpecialCharTok{/}\DecValTok{3}\NormalTok{)}\SpecialCharTok{*}\NormalTok{(x}\SpecialCharTok{\^{}}\NormalTok{(}\SpecialCharTok{{-}}\DecValTok{2}\SpecialCharTok{/}\DecValTok{3}\NormalTok{))}
\SpecialCharTok{+}\NormalTok{   out\_ }\OtherTok{\textless{}{-}}\NormalTok{ out\_}\SpecialCharTok{*}\NormalTok{(}\SpecialCharTok{{-}}\DecValTok{1}\NormalTok{)}
\SpecialCharTok{+}   \FunctionTok{return}\NormalTok{(out\_)}
\SpecialCharTok{+}\NormalTok{ \}}
\end{Highlighting}
\end{Shaded}

\hypertarget{q2}{%
\subsection{Q2:}\label{q2}}

\emph{Create a plot of f and f on {[}0,4{]} in different colors and line
types and add a legend.} \textbf{MY SOLUTION:}

\begin{Shaded}
\begin{Highlighting}[]
\SpecialCharTok{\textgreater{}} \CommentTok{\# write the original function with the name of f\_}
\ErrorTok{\textgreater{}}\NormalTok{ f\_ }\OtherTok{\textless{}{-}} \ControlFlowTok{function}\NormalTok{(x)\{}
\SpecialCharTok{+}\NormalTok{   out\_ }\OtherTok{\textless{}{-}}\NormalTok{ (}\SpecialCharTok{{-}}\DecValTok{1}\NormalTok{)}\SpecialCharTok{*}\FunctionTok{log}\NormalTok{(x}\SpecialCharTok{\^{}}\DecValTok{2} \SpecialCharTok{+} \DecValTok{1}\NormalTok{) }\SpecialCharTok{+}\NormalTok{ x}\SpecialCharTok{\^{}}\NormalTok{(}\DecValTok{1}\SpecialCharTok{/}\DecValTok{3}\NormalTok{)}
\SpecialCharTok{+}   \FunctionTok{return}\NormalTok{(out\_)\}}
\SpecialCharTok{\textgreater{}} \CommentTok{\# since this question requires the maximum, I use {-}1 times the function.}
\ErrorTok{\textgreater{}}\NormalTok{ f\_neg }\OtherTok{\textless{}{-}} \ControlFlowTok{function}\NormalTok{(x)\{}
\SpecialCharTok{+}\NormalTok{   out\_ }\OtherTok{\textless{}{-}}\NormalTok{ (}\SpecialCharTok{{-}}\DecValTok{1}\NormalTok{)}\SpecialCharTok{*}\FunctionTok{log}\NormalTok{(x}\SpecialCharTok{\^{}}\DecValTok{2} \SpecialCharTok{+} \DecValTok{1}\NormalTok{) }\SpecialCharTok{+}\NormalTok{ x}\SpecialCharTok{\^{}}\NormalTok{(}\DecValTok{1}\SpecialCharTok{/}\DecValTok{3}\NormalTok{)}
\SpecialCharTok{+}   \FunctionTok{return}\NormalTok{(out\_}\SpecialCharTok{*}\NormalTok{(}\SpecialCharTok{{-}}\DecValTok{1}\NormalTok{))\}}
\SpecialCharTok{\textgreater{}} 
\ErrorTok{\textgreater{}} \CommentTok{\# first plot the original function}
\ErrorTok{\textgreater{}}\NormalTok{ x }\OtherTok{\textless{}{-}} \FunctionTok{seq}\NormalTok{(}\DecValTok{0}\NormalTok{,}\DecValTok{4}\NormalTok{,}\FloatTok{0.01}\NormalTok{)}
\SpecialCharTok{\textgreater{}} \CommentTok{\# plot the original function with blue line}
\ErrorTok{\textgreater{}} \FunctionTok{plot}\NormalTok{(x, }\FunctionTok{f\_}\NormalTok{(x), }\AttributeTok{col=}\StringTok{"blue"}\NormalTok{, }\AttributeTok{type =} \StringTok{"l"}\NormalTok{, }\AttributeTok{ylim =} \FunctionTok{c}\NormalTok{(}\SpecialCharTok{{-}}\FloatTok{1.5}\NormalTok{,}\DecValTok{1}\NormalTok{))}
\SpecialCharTok{\textgreater{}} \CommentTok{\# plot the first derivative with red line}
\ErrorTok{\textgreater{}} \FunctionTok{lines}\NormalTok{(x, }\FunctionTok{f\_prime}\NormalTok{(x), }\AttributeTok{col=}\StringTok{"red"}\NormalTok{, }\AttributeTok{type =} \StringTok{"l"}\NormalTok{)}
\SpecialCharTok{\textgreater{}} \CommentTok{\# add the legend}
\ErrorTok{\textgreater{}} \FunctionTok{legend}\NormalTok{(}\DecValTok{3}\NormalTok{,}\DecValTok{1}\NormalTok{, }\AttributeTok{inset =} \FloatTok{0.1}\NormalTok{, }\FunctionTok{c}\NormalTok{(}\StringTok{"f\_"}\NormalTok{,}\StringTok{"f\_prime"}\NormalTok{), }\AttributeTok{lty =} \DecValTok{1}\NormalTok{, }
\SpecialCharTok{+}        \AttributeTok{col =} \FunctionTok{c}\NormalTok{(}\StringTok{"blue"}\NormalTok{,}\StringTok{"red"}\NormalTok{), }\AttributeTok{title=}\StringTok{"line Type"}\NormalTok{)}
\SpecialCharTok{\textgreater{}} \CommentTok{\# add a horizontal line to indicate the y=0}
\ErrorTok{\textgreater{}} \FunctionTok{abline}\NormalTok{(}\AttributeTok{h=}\DecValTok{0}\NormalTok{,}\AttributeTok{lty=}\DecValTok{3}\NormalTok{)}
\SpecialCharTok{\textgreater{}} \FunctionTok{abline}\NormalTok{(}\AttributeTok{v=}\FloatTok{0.3683525}\NormalTok{,}\AttributeTok{lty=}\DecValTok{3}\NormalTok{)}
\SpecialCharTok{\textgreater{}} \FunctionTok{mtext}\NormalTok{(}\StringTok{"Figure 1. The Lines of the Given Function and it\textquotesingle{}s First Derivative"}\NormalTok{,}
\SpecialCharTok{+}       \AttributeTok{side =} \DecValTok{3}\NormalTok{,}
\SpecialCharTok{+}       \AttributeTok{line =} \SpecialCharTok{{-}}\DecValTok{2}\NormalTok{,}
\SpecialCharTok{+}       \AttributeTok{outer =}\NormalTok{ T)}
\end{Highlighting}
\end{Shaded}

\includegraphics{Homework_05_Pan-Lei_files/figure-latex/unnamed-chunk-2-1.pdf}

\hypertarget{q3}{%
\subsection{Q3:}\label{q3}}

\emph{Finish the functions that I started in the R code notes for
univariate optimization for the golden section search, the bisection
method, and Newton's method.}

\textbf{MY SOLUTION:}\\
\#\#\# 3.1 The Golden Section Search

\begin{Shaded}
\begin{Highlighting}[]
\SpecialCharTok{\textgreater{}}\NormalTok{ golden }\OtherTok{\textless{}{-}} \ControlFlowTok{function}\NormalTok{(f, int, }\AttributeTok{precision =} \FloatTok{1e{-}6}\NormalTok{)}
\SpecialCharTok{+}\NormalTok{ \{}
\SpecialCharTok{+}\NormalTok{   rho }\OtherTok{\textless{}{-}}\NormalTok{ (}\DecValTok{3}\SpecialCharTok{{-}}\FunctionTok{sqrt}\NormalTok{(}\DecValTok{5}\NormalTok{))}\SpecialCharTok{/}\DecValTok{2} \CommentTok{\# ::: Golden ratio}
\SpecialCharTok{+}   \CommentTok{\# ::: Work out first iteration here}
\SpecialCharTok{+}\NormalTok{   f\_a }\OtherTok{\textless{}{-}} \FunctionTok{f}\NormalTok{(int[}\DecValTok{1}\NormalTok{] }\SpecialCharTok{+}\NormalTok{ rho}\SpecialCharTok{*}\NormalTok{(}\FunctionTok{diff}\NormalTok{(int)))}
\SpecialCharTok{+}\NormalTok{   f\_b }\OtherTok{\textless{}{-}} \FunctionTok{f}\NormalTok{(int[}\DecValTok{2}\NormalTok{] }\SpecialCharTok{{-}}\NormalTok{ rho}\SpecialCharTok{*}\NormalTok{(}\FunctionTok{diff}\NormalTok{(int)))}
\SpecialCharTok{+}   \DocumentationTok{\#\#\# How many iterations will we need to reach the desired precision?}
\SpecialCharTok{+}\NormalTok{   N }\OtherTok{\textless{}{-}} \FunctionTok{ceiling}\NormalTok{(}\FunctionTok{log}\NormalTok{(precision}\SpecialCharTok{/}\NormalTok{(}\FunctionTok{diff}\NormalTok{(int)))}\SpecialCharTok{/}\FunctionTok{log}\NormalTok{(}\DecValTok{1}\SpecialCharTok{{-}}\NormalTok{rho))}
\SpecialCharTok{+}   \ControlFlowTok{for}\NormalTok{ (i }\ControlFlowTok{in} \DecValTok{1}\SpecialCharTok{:}\NormalTok{(N))                    }\CommentTok{\# index the number of iterations}
\SpecialCharTok{+}\NormalTok{   \{}
\SpecialCharTok{+}     \ControlFlowTok{if}\NormalTok{ (f\_a }\SpecialCharTok{\textless{}}\NormalTok{ f\_b)  }
\SpecialCharTok{+}\NormalTok{     \{}
\SpecialCharTok{+}\NormalTok{       int[}\DecValTok{2}\NormalTok{] }\OtherTok{\textless{}{-}}\NormalTok{ int[}\DecValTok{2}\NormalTok{] }\SpecialCharTok{{-}}\NormalTok{ rho}\SpecialCharTok{*}\NormalTok{(}\FunctionTok{diff}\NormalTok{(int))}
\SpecialCharTok{+}\NormalTok{       f\_b }\OtherTok{\textless{}{-}} \FunctionTok{f}\NormalTok{(int[}\DecValTok{2}\NormalTok{])}
\SpecialCharTok{+}\NormalTok{     \} }\ControlFlowTok{else}\NormalTok{\{}
\SpecialCharTok{+}       \ControlFlowTok{if}\NormalTok{ (f\_a }\SpecialCharTok{\textgreater{}=}\NormalTok{ f\_b)}
\SpecialCharTok{+}\NormalTok{       \{}
\SpecialCharTok{+}\NormalTok{         int[}\DecValTok{1}\NormalTok{] }\OtherTok{\textless{}{-}}\NormalTok{ int[}\DecValTok{1}\NormalTok{] }\SpecialCharTok{+}\NormalTok{ rho}\SpecialCharTok{*}\NormalTok{(}\FunctionTok{diff}\NormalTok{(int))}
\SpecialCharTok{+}\NormalTok{         f\_a }\OtherTok{\textless{}{-}} \FunctionTok{f}\NormalTok{(int[}\DecValTok{1}\NormalTok{])}
\SpecialCharTok{+}\NormalTok{       \} \}}
\SpecialCharTok{+}\NormalTok{   \}}
\SpecialCharTok{+}\NormalTok{   int}
\SpecialCharTok{+}   \FunctionTok{print}\NormalTok{(}\FunctionTok{paste0}\NormalTok{(}\StringTok{"Iteration for "}\NormalTok{,N,}\StringTok{" times;"}\NormalTok{,}
\SpecialCharTok{+}                \StringTok{" The location is at "}\NormalTok{, }\FunctionTok{round}\NormalTok{(int[}\DecValTok{1}\NormalTok{],}\DecValTok{6}\NormalTok{)))}
\SpecialCharTok{+}\NormalTok{ \}}
\end{Highlighting}
\end{Shaded}

\hypertarget{the-bisection-method}{%
\subsubsection{3.2 The Bisection Method}\label{the-bisection-method}}

More information about this method can be found on \emph{Page 116} of
Chong and Zak (2013).

\begin{Shaded}
\begin{Highlighting}[]
\SpecialCharTok{\textgreater{}}\NormalTok{ bisection }\OtherTok{\textless{}{-}} \ControlFlowTok{function}\NormalTok{(f\_prime, int, }\AttributeTok{precision =} \FloatTok{1e{-}7}\NormalTok{)}
\SpecialCharTok{+}\NormalTok{ \{}
\SpecialCharTok{+}   \CommentTok{\# ::: f\_prime is the function for the first derivative}
\SpecialCharTok{+}   \CommentTok{\# ::: of f, int is an interval such as c(0,1) which }
\SpecialCharTok{+}   \CommentTok{\# ::: denotes the domain}
\SpecialCharTok{+}   
\SpecialCharTok{+}\NormalTok{   N }\OtherTok{\textless{}{-}} \FunctionTok{ceiling}\NormalTok{(}\FunctionTok{log}\NormalTok{(precision}\SpecialCharTok{/}\NormalTok{(}\FunctionTok{diff}\NormalTok{(int)))}\SpecialCharTok{/}\FunctionTok{log}\NormalTok{(.}\DecValTok{5}\NormalTok{))}
\SpecialCharTok{+}   \CommentTok{\# find the midpoint of the initial uncertainty range}
\SpecialCharTok{+}\NormalTok{   midpoint }\OtherTok{\textless{}{-}}\NormalTok{ (int[}\DecValTok{1}\NormalTok{]}\SpecialCharTok{+}\NormalTok{int[}\DecValTok{2}\NormalTok{]) }\SpecialCharTok{/}\DecValTok{2}
\SpecialCharTok{+}   \CommentTok{\# evaluate the f\_prime on the midpoint}
\SpecialCharTok{+}\NormalTok{   f\_prime\_a }\OtherTok{\textless{}{-}} \FunctionTok{f\_prime}\NormalTok{(midpoint)}
\SpecialCharTok{+}\NormalTok{   i }\OtherTok{\textless{}{-}} \DecValTok{1}
\SpecialCharTok{+}   \ControlFlowTok{for}\NormalTok{ (i }\ControlFlowTok{in} \DecValTok{1}\SpecialCharTok{:}\NormalTok{N)}
\SpecialCharTok{+}\NormalTok{   \{}
\SpecialCharTok{+}\NormalTok{     i }\OtherTok{\textless{}{-}}\NormalTok{ i }\SpecialCharTok{+} \DecValTok{1}
\SpecialCharTok{+}     \ControlFlowTok{if}\NormalTok{(f\_prime\_a }\SpecialCharTok{\textless{}} \DecValTok{0}\NormalTok{)}
\SpecialCharTok{+}\NormalTok{     \{}
\SpecialCharTok{+}       \CommentTok{\# if the f\_prime on the midpoint is less than 0,}
\SpecialCharTok{+}       \CommentTok{\# the minimizer must be on the right side of midpoint.}
\SpecialCharTok{+}\NormalTok{       int[}\DecValTok{1}\NormalTok{] }\OtherTok{\textless{}{-}}\NormalTok{ midpoint}
\SpecialCharTok{+}       \CommentTok{\# update the uncertainty range}
\SpecialCharTok{+}\NormalTok{       midpoint }\OtherTok{\textless{}{-}}\NormalTok{ (int[}\DecValTok{1}\NormalTok{]}\SpecialCharTok{+}\NormalTok{int[}\DecValTok{2}\NormalTok{]) }\SpecialCharTok{/}\DecValTok{2}
\SpecialCharTok{+}\NormalTok{     \} }\ControlFlowTok{else}
\SpecialCharTok{+}       \ControlFlowTok{if}\NormalTok{(f\_prime\_a }\SpecialCharTok{\textgreater{}} \DecValTok{0}\NormalTok{)}
\SpecialCharTok{+}\NormalTok{       \{}
\SpecialCharTok{+}         \CommentTok{\# if the f\_prime on the midpoint is less than 0,}
\SpecialCharTok{+}         \CommentTok{\# the minimizer must be on the left side of midpoint.}
\SpecialCharTok{+}\NormalTok{         int[}\DecValTok{2}\NormalTok{] }\OtherTok{\textless{}{-}}\NormalTok{ midpoint}
\SpecialCharTok{+}\NormalTok{         midpoint }\OtherTok{\textless{}{-}}\NormalTok{ (int[}\DecValTok{1}\NormalTok{]}\SpecialCharTok{+}\NormalTok{int[}\DecValTok{2}\NormalTok{]) }\SpecialCharTok{/}\DecValTok{2}
\SpecialCharTok{+}\NormalTok{       \} }\ControlFlowTok{else}
\SpecialCharTok{+}         \ControlFlowTok{if}\NormalTok{(f\_prime\_a }\SpecialCharTok{==} \DecValTok{0}\NormalTok{)}
\SpecialCharTok{+}\NormalTok{         \{}
\SpecialCharTok{+}           \ControlFlowTok{break}
\SpecialCharTok{+}\NormalTok{         \}}
\SpecialCharTok{+}     \CommentTok{\# ::: FILL IN CODE HERE (UPDATE)}
\SpecialCharTok{+}\NormalTok{     f\_prime\_a }\OtherTok{\textless{}{-}} \FunctionTok{f\_prime}\NormalTok{(midpoint)}
\SpecialCharTok{+}\NormalTok{   \}}
\SpecialCharTok{+}\NormalTok{   int}
\SpecialCharTok{+}   \FunctionTok{print}\NormalTok{(}\FunctionTok{paste0}\NormalTok{(}\StringTok{"Iteration for "}\NormalTok{,N,}\StringTok{" times;"}\NormalTok{,}
\SpecialCharTok{+}                \StringTok{" The location is at "}\NormalTok{, }\FunctionTok{round}\NormalTok{(int[}\DecValTok{1}\NormalTok{],}\DecValTok{6}\NormalTok{)))}
\SpecialCharTok{+}\NormalTok{ \}}
\end{Highlighting}
\end{Shaded}

\hypertarget{the-newtons-method}{%
\subsubsection{3.3 The Newton's Method}\label{the-newtons-method}}

More information about this method can be found on \emph{Page 116} of
Chong and Zak (2013).

\begin{Shaded}
\begin{Highlighting}[]
\SpecialCharTok{\textgreater{}}\NormalTok{ newton }\OtherTok{\textless{}{-}} \ControlFlowTok{function}\NormalTok{(f\_prime, f\_dbl, }\AttributeTok{precision =} \FloatTok{1e{-}6}\NormalTok{, start)}
\SpecialCharTok{+}\NormalTok{ \{}
\SpecialCharTok{+}\NormalTok{   x\_old }\OtherTok{\textless{}{-}}\NormalTok{ start}
\SpecialCharTok{+}\NormalTok{   x\_new }\OtherTok{\textless{}{-}}\NormalTok{ x\_old }\SpecialCharTok{{-}} \FunctionTok{f\_prime}\NormalTok{(x\_old)}\SpecialCharTok{/}\FunctionTok{f\_dbl}\NormalTok{(x\_old)}
\SpecialCharTok{+}   
\SpecialCharTok{+}\NormalTok{   i }\OtherTok{\textless{}{-}} \DecValTok{1}
\SpecialCharTok{+}   \FunctionTok{print}\NormalTok{(}\FunctionTok{paste0}\NormalTok{(}\StringTok{"Iteration "}\NormalTok{, i, }\StringTok{"; Estimate = "}\NormalTok{, x\_new) )}
\SpecialCharTok{+}   \ControlFlowTok{while}\NormalTok{ (}\FunctionTok{abs}\NormalTok{(x\_new}\SpecialCharTok{{-}}\NormalTok{x\_old) }\SpecialCharTok{\textgreater{}}\NormalTok{ precision)\{}
\SpecialCharTok{+}\NormalTok{     x\_old }\OtherTok{\textless{}{-}}\NormalTok{ x\_new}
\SpecialCharTok{+}\NormalTok{     x\_new }\OtherTok{\textless{}{-}}\NormalTok{ x\_old }\SpecialCharTok{{-}} \FunctionTok{f\_prime}\NormalTok{(x\_old)}\SpecialCharTok{/}\FunctionTok{f\_dbl}\NormalTok{(x\_old)}
\SpecialCharTok{+}     \CommentTok{\# ::: redefine variables and calculate new estimate}
\SpecialCharTok{+}     \CommentTok{\# ::: keep track of iteration history}
\SpecialCharTok{+}     \FunctionTok{print}\NormalTok{(}\FunctionTok{paste0}\NormalTok{(}\StringTok{"Iteration "}\NormalTok{, i}\SpecialCharTok{+}\DecValTok{1}\NormalTok{, }\StringTok{"; Estimate = "}\NormalTok{, x\_new) )}
\SpecialCharTok{+}\NormalTok{     i }\OtherTok{\textless{}{-}}\NormalTok{ i }\SpecialCharTok{+} \DecValTok{1}
\SpecialCharTok{+}\NormalTok{   \}}
\SpecialCharTok{+}\NormalTok{   x\_new}
\SpecialCharTok{+}\NormalTok{ \}}
\end{Highlighting}
\end{Shaded}

\hypertarget{q4}{%
\subsection{Q4:}\label{q4}}

\emph{Apply each of the three functions to this example to discover the
minimum. Keep track of and report the number of iterations required for
each method. Report the coordinates of the minimum discovered by each of
the three functions as well as the number of iterations required}

\textbf{MY SOLUTION:}

\hypertarget{apply-the-golden-section-search}{%
\subsubsection{4.1 Apply the Golden Section
Search}\label{apply-the-golden-section-search}}

\begin{Shaded}
\begin{Highlighting}[]
\SpecialCharTok{\textgreater{}} \FunctionTok{golden}\NormalTok{(f\_neg, }\FunctionTok{c}\NormalTok{(}\DecValTok{0}\NormalTok{,}\DecValTok{4}\NormalTok{))}
\NormalTok{[}\DecValTok{1}\NormalTok{] }\StringTok{"Iteration for 32 times;The location is at 0.368352"}
\end{Highlighting}
\end{Shaded}

\hypertarget{apply-the-bisection-method}{%
\subsubsection{4.2 Apply the Bisection
Method}\label{apply-the-bisection-method}}

\begin{Shaded}
\begin{Highlighting}[]
\SpecialCharTok{\textgreater{}} \FunctionTok{bisection}\NormalTok{(f\_prime\_neg, }\FunctionTok{c}\NormalTok{(}\DecValTok{0}\NormalTok{,}\DecValTok{4}\NormalTok{))}
\NormalTok{[}\DecValTok{1}\NormalTok{] }\StringTok{"Iteration for 26 times;The location is at 0.368352"}
\end{Highlighting}
\end{Shaded}

\hypertarget{apply-the-newtons-method}{%
\subsubsection{4.3 Apply the Newton's
Method}\label{apply-the-newtons-method}}

First, I need to get the second derivative of the original function.
Based on the Quotient Rule and Chain Rule, one can easily have
\[f(x)''= -6x^2-2-\frac{2}{9}x^{-\frac{5}{3}}\]

\begin{Shaded}
\begin{Highlighting}[]
\SpecialCharTok{\textgreater{}} \CommentTok{\# write the second derivative function}
\ErrorTok{\textgreater{}}\NormalTok{ f\_dbl }\OtherTok{\textless{}{-}} \ControlFlowTok{function}\NormalTok{(x)\{}
\SpecialCharTok{+}\NormalTok{   out\_ }\OtherTok{\textless{}{-}} \SpecialCharTok{{-}}\DecValTok{6}\SpecialCharTok{*}\NormalTok{x}\SpecialCharTok{\^{}}\DecValTok{2{-}2}\SpecialCharTok{{-}}\NormalTok{(}\DecValTok{2}\SpecialCharTok{/}\DecValTok{9}\NormalTok{)}\SpecialCharTok{*}\NormalTok{x}\SpecialCharTok{\^{}}\NormalTok{(}\SpecialCharTok{{-}}\DecValTok{5}\SpecialCharTok{/}\DecValTok{3}\NormalTok{)}
\SpecialCharTok{+}   \FunctionTok{return}\NormalTok{(out\_)}
\SpecialCharTok{+}\NormalTok{ \}}
\SpecialCharTok{\textgreater{}} \CommentTok{\# since this question requires the maximum, I use {-}1 times the function.}
\ErrorTok{\textgreater{}}\NormalTok{ f\_dbl\_neg }\OtherTok{\textless{}{-}} \ControlFlowTok{function}\NormalTok{(x)\{}
\SpecialCharTok{+}\NormalTok{   out\_ }\OtherTok{\textless{}{-}} \SpecialCharTok{{-}}\DecValTok{6}\SpecialCharTok{*}\NormalTok{x}\SpecialCharTok{\^{}}\DecValTok{2{-}2}\SpecialCharTok{{-}}\NormalTok{(}\DecValTok{2}\SpecialCharTok{/}\DecValTok{9}\NormalTok{)}\SpecialCharTok{*}\NormalTok{x}\SpecialCharTok{\^{}}\NormalTok{(}\SpecialCharTok{{-}}\DecValTok{5}\SpecialCharTok{/}\DecValTok{3}\NormalTok{)}
\SpecialCharTok{+}   \FunctionTok{return}\NormalTok{(out\_}\SpecialCharTok{*}\NormalTok{(}\SpecialCharTok{{-}}\DecValTok{1}\NormalTok{))}
\SpecialCharTok{+}\NormalTok{ \}}
\end{Highlighting}
\end{Shaded}

Plug the first and the second derivatives into the Newton's Method
function.

\begin{Shaded}
\begin{Highlighting}[]
\SpecialCharTok{\textgreater{}} \FunctionTok{newton}\NormalTok{(f\_prime\_neg, f\_dbl\_neg,}\AttributeTok{start =} \DecValTok{1}\NormalTok{)}
\NormalTok{[}\DecValTok{1}\NormalTok{] }\StringTok{"Iteration 1; Estimate = 0.918918918918919"}
\NormalTok{[}\DecValTok{1}\NormalTok{] }\StringTok{"Iteration 2; Estimate = 0.830999851550479"}
\NormalTok{[}\DecValTok{1}\NormalTok{] }\StringTok{"Iteration 3; Estimate = 0.736988742573808"}
\NormalTok{[}\DecValTok{1}\NormalTok{] }\StringTok{"Iteration 4; Estimate = 0.639870150729756"}
\NormalTok{[}\DecValTok{1}\NormalTok{] }\StringTok{"Iteration 5; Estimate = 0.546642635261104"}
\NormalTok{[}\DecValTok{1}\NormalTok{] }\StringTok{"Iteration 6; Estimate = 0.468667930513546"}
\NormalTok{[}\DecValTok{1}\NormalTok{] }\StringTok{"Iteration 7; Estimate = 0.416016821629135"}
\NormalTok{[}\DecValTok{1}\NormalTok{] }\StringTok{"Iteration 8; Estimate = 0.388211559563474"}
\NormalTok{[}\DecValTok{1}\NormalTok{] }\StringTok{"Iteration 9; Estimate = 0.376060038291039"}
\NormalTok{[}\DecValTok{1}\NormalTok{] }\StringTok{"Iteration 10; Estimate = 0.371254711645403"}
\NormalTok{[}\DecValTok{1}\NormalTok{] }\StringTok{"Iteration 11; Estimate = 0.369432581306114"}
\NormalTok{[}\DecValTok{1}\NormalTok{] }\StringTok{"Iteration 12; Estimate = 0.368752709077139"}
\NormalTok{[}\DecValTok{1}\NormalTok{] }\StringTok{"Iteration 13; Estimate = 0.368500562445562"}
\NormalTok{[}\DecValTok{1}\NormalTok{] }\StringTok{"Iteration 14; Estimate = 0.368407257199536"}
\NormalTok{[}\DecValTok{1}\NormalTok{] }\StringTok{"Iteration 15; Estimate = 0.368372758807196"}
\NormalTok{[}\DecValTok{1}\NormalTok{] }\StringTok{"Iteration 16; Estimate = 0.36836000738855"}
\NormalTok{[}\DecValTok{1}\NormalTok{] }\StringTok{"Iteration 17; Estimate = 0.368355294697857"}
\NormalTok{[}\DecValTok{1}\NormalTok{] }\StringTok{"Iteration 18; Estimate = 0.36835355304667"}
\NormalTok{[}\DecValTok{1}\NormalTok{] }\StringTok{"Iteration 19; Estimate = 0.368352909401224"}
\NormalTok{[}\DecValTok{1}\NormalTok{] }\FloatTok{0.3683529}
\end{Highlighting}
\end{Shaded}

All three methods have the identical results, that is the maximum point
is at \(x=.368352\). As for the iteration times at the x interval
\([0,4]\), the Golden Section Search iterated 32 times; the Bisection
Method iterated 26 times. However, Newton's method iteration time
largely depends on the start numbers. That is, the more guessing number
close to the target, the less times needed.

\end{document}
